%! TEX root = **/010-main.tex
% vim: spell spelllang=ca :

\section{Experiments}%
\label{sec:experiments}

Tal com s'ha explicat a les seccions anteriors, van dur a terme diversos experiments per escollir quins descriptors
i classificadors utilitzar. Per dur a terme els experiments, vam dividir les imatges en dos conjunts en proporció 1:4.
Com que teníem series de imatges i no imatges individuals, per tal de no entrenar i desprès fer els test amb imatges
de la mateixa sèrie, la divisió la vam fer per series senceres.

Així doncs, teníem un 20\% de els series d'imatges com a imatges de test i l'altre 80\% per a entrenar. El problema d'aquest
mètode es que no es considera el nombre d'imatges de les series i per tant pot haver series amb moltes imatges que
afectin al classificador tant al entrenar com al avaluar-lo.

Al entrenar els classificadors vam fer 5 \emph{folds} de \emph{cross-validation} per evitar overfitting del model. Per
avaluar els models ens vam centrar en l'accuracy d'aquests sobre el conjunt de \emph{test}.

La primera versió del classificador vam utilitzar descriptors només basats en les propietats de les àrees mes grans
de cada color. Amb \emph{weighted KNN} obteníem una precisió de 46.3\% i SVM del 32.8\%. Analitzant les segmentacions
de color d'algunes de les imatges que no classificàvem be, vam veure que el fons es confonia molt sovint amb els colors
i les formes no es reconeixien. A mes a mes, fèiem \texttt{imclearborder} per eliminar els elements que tocaven les vores
però això feia que molt sovint no quedes res. Per solucionar-ho, vam fer una segmentació del fons mes complexa en la
que busquem cercles i línies (la descripció en detall esta explicada a la \cref{sub:seg_fons}.

També vam afegir altres descriptors basats en Hough que buscaven rectes i cercles per poder distingir millor entre
triangles, cercles i altres formes.

Provant el classificador en grups mes generals de senyals (agrupant cercles vermells, triangles, cercles blaus \dots)
obteníem accuracy del 90\% amb SVM lineal. Tot i no ser un molt bon resultat considerant que les classes eren molt
genèriques, ens indica que els descriptors que teníem eren suficients per distingir les formes de les senyals i colors en la majoria de casos.

Amb la nova segmentació els resultats van millorar lleugerament, però seguíem sense poder distingir entre els nombres
i dibuixos de dintre les senyals. Per a distingir-los, vam afegir un nou descriptor aplicant HOG
(explicat en detall a la \cref{sub:HOG}).

El problema amb aquest nou descriptor es que s'ha de computar per blanc i per negre i te 900 variables. Pel que passàvem
a tenir casi 2000 descriptors per imatge que trigaven bastant en extreure'ls de totes les imatges i a entrenar.
Per assolir temps d'entrenament assequibles necessitàvem un mostreig del 10\%
El problema es que amb 10\% agafem només unes 3000 imatges tenint 2000
descriptors fent que els resultats fossin molt poc precisos. El mateix passava
al reduir la mida dels descriptors de HOG o fent una selecció dels descriptors
mes rellevants.

Per continuar fent els experiments vam dividir el classificador en dues etapes: identificació de la senyal en termes generals
i identificació de la senyal concreta en funció del dibuix utilitzant HOG. D'aquesta manera podíem millorar els descriptors
de HOG sense haver de tornar a entrenar el primer classificador i viceversa. El problema d'aquest sistema es que l'error
del primer classificador s'acumula al següent, per tant, si el primer no es molt
fiable es perjudicial.

A mesura que afegíem descriptors, el KNN que abans donava millors resultats que SVM començava a veure afectat per
descriptors irrellevants i donava pitjors resultats. D'entre els diferents SVM, el millor resultat l'obteníem amb
el SVM quadràtic i one-vs-all, però tenint casi 2000 features no podíem fer one-vs-all ja que trigava massa.

Utilitzàvem un primer classificador que distingia entre les categories
principals:

\begin{table}[H]
    \centering
    \caption{Taula de categories principals}%
    \label{tab:cat_prin}
    \begin{tabular}{lc}
        \toprule
        Categoria & Senyals que conté \\
        \midrule
        Cercles blanc i vermells amb dibuix a dintre & [0:5, 7,8, 16] \\
        Cercles blancs amb dibuix & [6, 32, 41, 42] \\
        Cercles prohibició d'avançar & [9, 10] \\
        Calcada amb prioritat & [12] \\
        Ceda & [13] \\
        Stop & [14] \\
        Circulació prohibida & [14] \\
        Direcció prohibida & [17] \\
        Triangles & [11, 18:31] \\
        Cercles blaus & [33:40] \\
        \bottomrule
    \end{tabular}
\end{table}

Tenim 5 categories úniques i cinc que s'han de refinar utilitzant els altres
classificadors. Vam entrenar un classificador per cada una de les 5 categories
que no distingíem directament al primer classificador utilitzant descriptors de
HOG. En tots els classificadors utilitzem SVM quadràtic (one-vs-one) perquè es
el que donava millors resultats.

