%! TEX root = **/010-main.tex
% vim: spell spelllang=ca :

% DESCRIPTORS
% RegionProps -> Circularity
%


%- Justificació del descriptors utilitzats i procediment d'obtenció.
%- Descripció de les rutines utilitzades, tant les implementades pels estudiants com les contribuïdes per altres
%- Classificador que s'utilitzarà. Motius.
%- Resultats preliminars (si n'hi ha algun)

\section{Descriptors Utilitzats}%
\label{sec:desc}

Per poder identificar les formes dels signes utilitzarem els següents descriptors per els diferents
colors rellevants de la imatge (blanc, negre, vermell, blau i groc). A partir d'aquests descriptors
el nostre classificador hauria de ser capaç d'identificar els signes.

\subsection{Circularity}

Aquest descriptor ens dona una mesura de la similitud de la figura a un cercle. Es calcula a partir
de la relació entre àrea i perímetre:

\[
\frac{4\pi \cdot \text{Area}}{\text{Perimeter}^2}
\]

\subsection{Euler Number}

Aquest descriptor ens diu el nombre de forats que conte la forma. Hauria de permetre identificar entre
altres el nombre de dígits d'una senyal.

\subsection{Extent}

L'Extent indica el percentatge de píxels que formen part de la forma dins del rectangle que l'encapsula.

\subsection{Shape Signature}

Calculant la distancia entre el centroide i els píxels del perímetre de la figura obtenim la
\emph{shape signature} de la qual podem obtenir dos indicadors:

\paragraph{Nombre de pics} Contant els nombre de màxims locals de la signatura obtenim el nombre de pics.

\paragraph{Diferencia entre màxim i mínim} La diferencia entre la distancia major i menor també serveix com a indicador de la forma.
