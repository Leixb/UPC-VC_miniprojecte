%! TEX root = **/010-main.tex
% vim: spell spelllang=ca :

% DESCRIPTORS
% RegionProps -> Circularity
%


%- Justificació del descriptors utilitzats i procediment d'obtenció.
%- Descripció de les rutines utilitzades, tant les implementades pels estudiants com les contribuïdes per altres
%- Classificador que s'utilitzarà. Motius.
%- Resultats preliminars (si n'hi ha algun)

\section{Segmentació}%
\label{sec:segmentacio}

Abans d'extreure els descriptors primer hem de fer segmentació per separar el fons de la
senyal i per separar els diversos colors que ens interessen.

\subsection{Colors}

Segmentem la imatge en cinc colors: blanc, negre, vermell, blau i groc. 
Primer de tot convertim la imatge a espai de color HSV per poder treballar amb més
facilitat. Després, trobem la saturació mitjana i ens quedem amb els píxels que
estan per sota, i calculem el valor mitjà d'aquells píxels. 

Les regions blanques
amb les que ens quedarem són aquelles amb un valor superior o igual a la mitja i una
saturació inferior a la mitja. Per les regions negres ens fixem en que el valor
sigui inferior a la mitja. 

Per la resta de colors, mesurem la distància del color de
cada píxel als colors i el color que estigui més a prop se'l queda com a part de les
seves regions.

\subsection{Fons}%
\label{sub:seg_fons}

\subsubsection{Cercles}

% Hough (imfindcircles)

\subsubsection{Triangles i altres}

% TODO
% imreconstruct
% bwconvexhull
% validacio de mascara


\section{Descriptors Utilitzats}%
\label{sec:desc}

En aquesta secció presentarem els descriptors que hem utilitzat finalment per
representar les imatges. Separem els descriptors en dos parts clares, els que descriuen
la forma del senya (cercle, triangle i màscara) i els que descriuen el contingut del
senyal, per colors.
A més, cal esmentar que hem provat un
munt d'altres descriptors però no han funcionat tant bé, com ara la \emph{Circularity}.

% explicar que vam fer diverses proves

\subsection{Descriptors Triangles}

Apliquem la transformada de Hough per buscar línies en angles de 90, 30 i 120 graus respecte la vertical. Sempre trobarem una línia per cada angle i serà aquella que passi 
per més punts dels trobats per Hough. 
Per cada línia extraiem 4 descriptors (en total 12):

\subsubsection{Longitud}

És a dir la longitud de la línia trobada normalitzada per l'amplada de la imatge.

\subsubsection{Distancia a la cantonada (vertical i horitzontal)}

La distàncies vertical i horitzontal del punt mig de la línia a la cantonada 
superior esquerra normalitzades per l'amplada de la imatge.

\subsubsection{H}

Fem servir H com a indicador de la qualitat de la línia. H representa el nombre de
punts trobats per Hough que pertanyen a la línia i també normalitzat per l'amplada 
de la imatge.

% tot dividit per w

\subsection{Descriptors Cercles}

Pels cercles busquem la seva presencia amb la funció \texttt{imfindcircles}.
Aquesta funció fa servir una Transformada de Hough Circular per trobar cercles
i ens retorna tant el centre com el radi. Fem servir la opció \emph{Sensitivity}
per augmentar-la fins a $0.9$, d'aquesta l'algoritme és menys fràgil davant
parts ocluses o més fosques. Si bé és cert que augmentar la \emph{Sensitivity}
pot comportar un augment de falsos positius, hem trobat que no era el cas per
les imatges que hem fet servir. A més, busquem tant cercles clars com foscos.

\subsubsection{radi}

El radi del cercle trobat ajustat a l'amplada de la imatge.

\subsubsection{Distancia al centre de la imatge}

La distància euclidiana des del centroide del cercle al centre de la imatge,
normalitzada per l'amplada de la imatge.

\subsection{Descriptors Mascara}

Calculant la distancia entre el centroide i els píxels del perímetre de la màscara
obtenim la
\emph{shape signature} de la qual podem obtenir dos indicadors:

\subsubsection{Nombre de pics}
Contant els nombre de màxims locals de la signatura obtenim el nombre de pics.

\subsubsection{Diferencia entre màxim i mínim}
La diferencia entre la distancia major i menor també serveix com a indicador de la forma.

\subsection{Descriptors Colors}%
\label{sub:desc_col}

Per poder identificar les formes dels signes utilitzarem els següents descriptors per els diferents
colors rellevants de la imatge (blanc, negre, vermell, blau i groc). Pels colors blanc i
negre agafem les 4 regions més grans i per la resta selecciones només la regió més 
gran. A partir d'aquests descriptors
de cada regió el nostre classificador hauria de ser capaç d'identificar els signes.

\subsubsection{CentroidDistance}

La distància euclidiana del centroide de la regió al centre de la imatge ajustada
per l'amplada de la imatge.

\subsubsection{Solidity}

La \emph{Solidity} representa la densitat de la regió i es calcula dividint l'àrea
de la regió per l'àrea del \emph{convex hull}.

\subsubsection{Eccentricity}

El rati entre el l'eix major i l'eix menor de l'e\lgem ipse que s'ajusta a la regió.

\subsubsection{EulerNumber}

Aquest descriptor ens diu el nombre de forats que conte la forma. Hauria de permetre identificar entre
altres el nombre de dígits d'una senyal.

\subsubsection{Extent}

L'Extent indica el percentatge de píxels que formen part de la forma dins del rectangle que l'encapsula.

\subsubsection{Coverage}

El rati de l'àrea de la regió respecte l'àrea de la senyal sencera.

\subsection{HOG}%
\label{sub:HOG}

Per diferenciar entre les diverses formes i dibuixos de les senyals, no hi havia prou amb els descriptors basats en
propietats geomètriques explicats a \cref{sub:desc_col}. Com que en general les imatges estaven orientades verticalment
i amb poca distorci\'o de perspectiva, vam decidir utilitzar un histograma de gradients orientats (HOG).


Per tenir sempre el mateix nombre de valors, vam redimensionar les imatges a un valor fix de 50x50 píxels amb
una mida de cela de 8x8, extraient  900 valors per imatge. Vam fer dos HOG, un sobre el blanc i un altre sobre el negre,
d'aquesta manera podíem 

%TODO

L'histograma el vam aplicar sobre el color negre i sobre el color
blanc, ja que son els dos colors que ens interessen per distingir les figures dins de les senyals.

%%%%%%%%%%%%%%%%%%%%%%%%%%%%%%%%5

